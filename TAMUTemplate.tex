%%%%%%%%%%%%%%%%%%%%%%%%%%%%%%%%%%%%%%%%%%%%%%%%%%%
%
%  New template code for TAMU Theses and Dissertations starting Fall 2016.  
%
%
%  Author: Sean Zachary Roberson
%  Version 3.17.01
%  Last Updated: 1/10/2017
%
%%%%%%%%%%%%%%%%%%%%%%%%%%%%%%%%%%%%%%%%%%%%%%%%%%%

\documentclass[12pt]{report}

%These next lines change the font. Fixes for certain
%fonts will be implemented in a future release.

%Comment this line if you do not wish to use Times
%New Roman. The font used will then be the LaTeX
%default of Computer Modern.
\usepackage{times}
%\usepackage{cmbright}
\usepackage[T1]{fontenc}

%Do not change these settings. The geometry package
%Adjusts the margins to those specified by the Thesis
%Manual. 
\usepackage[letterpaper]{geometry}
\geometry{verbose,tmargin=1.25in,bmargin=1.25in,lmargin=1.4in,rmargin=1.15in}
 \usepackage[doublespacing]{setspace}
 \usepackage{tocloft}
 \usepackage[rm, tiny, center, compact]{titlesec}
 \usepackage{indentfirst}
 \usepackage{etoolbox}

\usepackage{tocvsec2}
 \usepackage[titletoc]{appendix}
 \usepackage{appendix}
 \usepackage{tamuconfig}

\usepackage{rotating}

%These are common AMS packages. Many LaTeX documents
%have these packages declared in their preambles.
\usepackage{amsmath, amsthm}

%This package allows for the use of graphics in the
%document.
\usepackage{graphicx}

%If you have JPEG format images, add .jpg as an
%allowed file extension below. Same for Bitmaps (.bmp).
\DeclareGraphicsExtensions{.png}

%It is best practice to keep all your pictures in
%one folder inside the main directory in which your
%TeX file is kept. Here the folder is named "graphic."
%Replace the name here with your folder's name, if needed.
%The period is needed due to relative referencing.
\graphicspath{ {./graphic/} }

%If needed, this will allow you to add the word "Page"
%to extra pages on your front matter lists.
\usepackage{afterpage}

%This is from the mdwtools package; it fixes some
%footnote commands and allows you to have footnotes in
%tables via the savenotes environment.
\usepackage{footnote}

%If you have a table that spans multiple pages, this
%package allows for the use of the longtable
%environment.
\usepackage{longtable}

%For the Nomenclature. Test.
\usepackage{setspace}

%%%%%%%%%%%%%%%%%%%%%%%%%%%%%%%%%%%%%%%%%%%%%%%%%%%%%%%%%
%Please place all your personal packages here. Check to
%see if the packages you wish to use are not already
%declared above. Placing all your personal packages
%here allows me to determine if there are any package
%issues in compilation, as well as any conflicts
%that may arise by the order of loading.
%--Sean Zachary Roberson
%%%%%%%%%%%%%%%%%%%%%%%%%%%%%%%%%%%%%%%%%%%%%%%%%%%%%%%%%
%%%%%%%%%%%%%%%%%%%%%%%%%%%%%%%%%%%%%%%%%%%%%%%%%%%%%%%%%
%Begin student defined packages.
%%%%%%%%%%%%%%%%%%%%%%%%%%%%%%%%%%%%%%%%%%%%%%%%%%%%%%%%%


%%%%%%%%%%%%%%%%%%%%%%%%%%%%%%%%%%%%%%%%%%%%%%%%%%%%%%%%%
%End student defined packages.
%%%%%%%%%%%%%%%%%%%%%%%%%%%%%%%%%%%%%%%%%%%%%%%%%%%%%%%%%

% Added to fix issues with pdf searching in some versions of LaTeX
%\usepackage[T1]{fontenc}\usepackage{lmodern}
%%%%%%%%%%%%%%%%%%%%%%%%%%%%%

% Hyperref setup below.  You should be able to get away with using uncommenting just the first line.
%\usepackage[hidelinks]{hyperref}

% if \usepackage[hidelinks]{hyperref} doesn't work try this.
% \usepackage{hyperref}  % Hidelinks is an option that removes link visiability.  TAMU Thesis Offices prefers to not see the links. But often doesn't work.  
% 
% \hypersetup{
%     colorlinks=true,
%     linkcolor=black,
%     citecolor=black,
%     filecolor=black,
%     urlcolor=black,
% }
%%%%%%%  End of hyperref setup.  One of these two options should work, but my motto with hyperref is when in doubt, comment it out!
%%%%%%%%%  This hopefully fixes the problem with vertical spacing of section headings at the top of the page..  Commented out in 1.0.7
% \preto\section{%
% \ifnum\value{section}>0\addtocontents{toc}{\vskip-6pt}\fi
% }
% \preto\subsection{%
% \ifnum\value{subsection}=0\addtocontents{toc}{\vskip-6pt}\fi
% \ifnum\value{subsection}>0\addtocontents{toc}{\vskip-6pt}\fi
% } 
%%%%%%%%%%%%%%%%%%%%%%%%%%%%%%%%%%%%%%%%%%%%%%%%%%%%%%

%Test for list of algorithms.

%\newcommand{\listalgorithmname}{List of Algorithms}
%
%\newlistof{algorithm}{alg}{\listalgorithmname}
%
%\newcommand{\algorithm}[1]{
%\refstepcounter{algorithm}
%\par\noindent\textbef{X \thealgorithm. #1}
%\addcontentsline{algorithm}{algorithm}
%{\protext\numberline{\thesection. \thealgorithm}#1}\par}

\begin{document}

%The title of your document goes here.
%Spacing may need to be adjusted if your title is long
%and pushes the copyright off the page.
\renewcommand{\tamumanuscripttitle}{The Title of Your Thesis or Dissertation Goes In This Space To Let Us Know What Your Document is About}

%Type only Thesis, Dissertation, or Record of Study.
\renewcommand{\tamupapertype}{Thesis}

%Your full name goes here, as it is in university records. Check your student record on Howdy if there is any mismatch.
\renewcommand{\tamufullname}{Aggie D. Student}

%The degree title goes here. See the OGAPS site for more info.
\renewcommand{\tamudegree}{Master of Science}
\renewcommand{\tamuchairone}{Chair Name}


% Uncomment out the next line if you have co-chairs.  You will also need to edit the titlepage.tex file.
%\newcommand{\tamuchairtwo}{Additional Chair Name}
\renewcommand{\tamumemberone}{Committee Member 1}
\newcommand{\tamumembertwo}{Committee Member 2}
\newcommand{\tamumemberthree}{Committee Member 3}
\renewcommand{\tamudepthead}{Head of Department}

%Type only May, August, or December.
\renewcommand{\tamugradmonth}{December}
\renewcommand{\tamugradyear}{2017}
%Your department name goes here.
\renewcommand{\tamudepartment}{Mathematics}


\include{data/titlepage} % This is simply a file that formats and adds your titlepage, please do not edit this unless you have a specific need. .
\include{data/abstract}
%%%%%%%%%%%%%%%%%%%%%%%%%%%%%%%%%%%%%%%%%%%%%%%%%%%
%
%  New template code for TAMU Theses and Dissertations starting Fall 2016.  
%
%  Author: Sean Zachary Roberson
%	 Version 3.17.01
%  Last updated 1/10/2017
%
%%%%%%%%%%%%%%%%%%%%%%%%%%%%%%%%%%%%%%%%%%%%%%%%%%%

%%%%%%%%%%%%%%%%%%%%%%%%%%%%%%%%%%%%%%%%%%%%%%%%%%%%%%%%%%%%%%%%%%%%%%
%%                           DEDICATION
%%%%%%%%%%%%%%%%%%%%%%%%%%%%%%%%%%%%%%%%%%%%%%%%%%%%%%%%%%%%%%%%%%%%%
\chapter*{DEDICATION}
\addcontentsline{toc}{chapter}{DEDICATION}  % Needs to be set to part, so the TOC doesnt add 'CHAPTER ' prefix in the TOC.



\begin{center}
\vspace*{\fill}
To my mother, my father, my grandfather, and my grandmother. To see what happens with multiple lines, I extend this next part into a second line.
\vspace*{\fill}
\end{center}

\pagebreak{}

\include{data/acknowledgements}
\include{data/contributors}
\include{data/nomenclature}

\include{data/lists}  % This is simply a file that formats and adds your toc, lof, and lot, please do not edit this unless you have a specific need.

\include{data/section1}
\include{data/section2}
%%%%%%%%%%%%%%%%%%%%%%%%%%%%%%%%%%%%%%%%%%%%%%%%%%%
%
%  New template code for TAMU Theses and Dissertations starting Fall 2016.
%
%  Author: Sean Zachary Roberson
%	 Version 3.17.01
%  Last updated 1/10/2017
%
%%%%%%%%%%%%%%%%%%%%%%%%%%%%%%%%%%%%%%%%%%%%%%%%%%%
%%%%%%%%%%%%%%%%%%%%%%%%%%%%%%%%%%%%%%%%%%%%%%%%%%%%%%%%%%%%%%%%%%%%%%
%%                           SECTION III
%%%%%%%%%%%%%%%%%%%%%%%%%%%%%%%%%%%%%%%%%%%%%%%%%%%%%%%%%%%%%%%%%%%%%



\chapter{VERY, VERY, VERY LONG TITLE THAT FLOWS INTO A SECOND LINE FOR THE SAKE OF EXAMPLE}

Notice that the title of this section is long - much longer than the others. When you have long section titles, this template takes care of double spacing the lines in the title. If the title is long to fit in the table of contents, the template will single space the title.

\section{Yet Another Table}

Another table is placed here to show the effect of having tables in multiple sections. The list of tables should still double space between table titles, while single spacing long table titles.

%Fix table labeling.
\begin{table}[h!]
	\centering
	\begin{tabular}{|l|l|}
		\hline
		Dates & Attendance  \\ \hline
		August 8-10, 2008 & 3,523  \\ \hline
		August 14-16, 2009 & 4,003 \\ \hline
		July 9-11, 2010 & 5,049 \\ \hline
		August 5-7, 2011 & 6,891  \\ \hline
		August 10-12, 2012 & 9,464  \\ \hline
		August 16-18, 2013 & 11,077  \\ \hline
		July 18-20, 2014 & 14,686 \\ \hline
		July 31-August 2, 2015 & 18,411  \\ \hline
	\end{tabular}
	\caption{San Japan attendance. Data is taken from \cite{ANCONS}. I intentionally make the title of this table long so the single space effect is seen in the list of tables.}
\end{table}

You may be wondering why San Japan was chosen. There are a few reasons as to why I did this:

\begin{enumerate}
\item It is one of the fastest-growing anime conventions in Texas.
\item Filler.
\item I wanted a good variety of table examples.
\item Because conventions are cool.
\end{enumerate}

The \textit{enumerate} environment was used to generated an ordered list above.

\section{Section Test Example}
We insert another figure here, just for kicks.

\begin{figure}[h!]
	\centering
	\includegraphics[scale=0.5]{LowPass_Filter_Design.png}
	\caption{A low pass filter design.}
\end{figure}

\subsection{Filler, Filler, Filler}

This section has filler text.

\subsection{Subsection Test Example}
Test subsection for TOC display

\subsection{Subsection Test Example 2}
Test subsection for TOC display

\subsection{Section Summary}
  
This holds the summary.

\section{Section Test Example 3}
Test section for toc display only

\subsection{Subsection Test 1}
Test subsection for toc display only.

\subsection{Subsection Test 2}
Test subsection for toc display only.

%%%%%%%%%%%%%%%%%%%%%%%%%%%%%%%%%%%%%%%%%%%%%%%%%%%
%
%  New template code for TAMU Theses and Dissertations starting Fall 2016.
%
%  Author: Sean Zachary Roberson
%	 Version 3.17.01
%  Last updated 1/10/2017
%
%%%%%%%%%%%%%%%%%%%%%%%%%%%%%%%%%%%%%%%%%%%%%%%%%%%
%%%%%%%%%%%%%%%%%%%%%%%%%%%%%%%%%%%%%%%%%%%%%%%%%%%%%%%%%%%%%%%%%%%%%%
%%                           SECTION IV
%%%%%%%%%%%%%%%%%%%%%%%%%%%%%%%%%%%%%%%%%%%%%%%%%%%%%%%%%%%%%%%%%%%%%



\chapter{SUMMARY AND CONCLUSIONS \label{cha:Summary}}

**Some text/figure here**

\section{Challenges}
Section here is to test toc display only.

\section{Further Study}
Section here is to test toc display only.


%The next line is the format for inserting new sections.
%Replace the name "newsection"  with the name of your
%new section file.
%\include{data/newsection}

%fix spacing in bibliography, if any...
%%%%%%%%%%%%%%%%%%%%%%%%%%%%%%%%%%%%%%%%%%%%%%%%%%%%%%%%%%%%%
\let\oldbibitem\bibitem
\renewcommand{\bibitem}{\setlength{\itemsep}{0pt}\oldbibitem}
%%%%%%%%%%%%%%%%%%%%%%%%%%%%%%%%%%%%%%%%%%%%%%%%%%%%%%%%%%%%%%%
%The bibliography style declared is the IEEE format. If
%you require a different style, see the document
%bibstyles.pdf included in this package. This file,
%hosted by the University of Vienna, shows several
%bibliography styles and examples of in-text citation
%and a references page.
\bibliographystyle{ieeetr}

\phantomsection
\addcontentsline{toc}{chapter}{REFERENCES}

\renewcommand{\bibname}{{\normalsize\rm REFERENCES}}

%This file is a .bib database that contains the sources.
%This removes the dependency on the previous file
%bibliography.tex.
\bibliography{data/myReference}




%This next line includes appendices. The file
%appendix.tex contains commands pointing to
%the appendix files; be sure to change these
%pointers if you end up changing the filenames.
%Leave this commented if you will not need
%appendix material.
\include{data/appendices}


\end{document}
